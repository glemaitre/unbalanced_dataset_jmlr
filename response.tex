%%%%%%%%%%%%%%%%%%%%%%%%%%%%%%%%%%%%%%%%%
% Thin Formal Letter
% LaTeX Template
% Version 1.11 (8/12/12)
%
% This template has been downloaded from:
% http://www.LaTeXTemplates.com
%
% Original author:
% WikiBooks (http://en.wikibooks.org/wiki/LaTeX/Letters)
%
% License:
% CC BY-NC-SA 3.0 (http://creativecommons.org/licenses/by-nc-sa/3.0/)
%
%%%%%%%%%%%%%%%%%%%%%%%%%%%%%%%%%%%%%%%%%

% ----------------------------------------------------------------------------------------
% DOCUMENT CONFIGURATIONS
% ----------------------------------------------------------------------------------------

\documentclass{letter}

% Adjust margins for aesthetics
\addtolength{\voffset}{-0.5in}
\addtolength{\hoffset}{-0.3in}
\addtolength{\textheight}{2cm}

% \longindentation=0pt % Un-commenting this line will push the closing "Sincerely," to the left of the page

% ----------------------------------------------------------------------------------------
% YOUR NAME & ADDRESS SECTION
% ----------------------------------------------------------------------------------------

\signature{G. Lemaitre, F. Nogueira, and C. Aridas} % Your name for the signature at the bottom

\address{1 Rue Honor\'e d'Estienne d'Orves \\ 91120 Palaiseau \\ +33(0)761104782} % Your address and phone number

% ----------------------------------------------------------------------------------------

\begin{document}

% ----------------------------------------------------------------------------------------
% ADDRESSEE SECTION
% ----------------------------------------------------------------------------------------

\begin{letter}{Geoff Holmes \\ Action editor of JMLR} % Name/title of the addressee

  % ----------------------------------------------------------------------------------------
  % LETTER CONTENT SECTION
  % ----------------------------------------------------------------------------------------

  \opening{\textbf{Dear Sir,}}

  Alongside with the submission 16-365 at the MLOSS of the JMLR, we
  would like to thank the reviewers for their pertinent suggestions in
  an effort to improve our software.
  Therefore, we addressed the issues raised by the reviewers by
  improving the manuscript, the documentation, and the software.

  Below, we go into details regarding the changes which has been done:

  \textbf{Reviewer \#1:}

  \begin{itemize}
  \item \textit{I believe that a real world example of this package
      improving generalization performance would make the paper more
      convincing.}\\
    Due to the page limits, we did not added any real-world example in
    the manuscript itself. However, examples have been added in the
    documentation of the software. Additionally, we continuously aim
    at adding real-world case studies.
  \item \textit{Typos / phrasing suggestions.}\\
    We addressed the typos and adopted the phrasing suggested.
  \end{itemize}

  \textbf{Reviewer \#2:}

  \begin{itemize}
  \item \textit{A table could be added with the implemented methods
      grouped by type (over-sampling, under-sampling ...). The table
      could also include a categorization of the methods concerning
      the possibility of application only to two-class or multi-class
      problems. This could provide a better overview of the
      implemented methods.}\\
    We simplified the description for each method and add the
    suggested table indicating on which type of problems ---
    binary \emph{vs.} multi-class --- the methods can be used.
  \item \textit{It would help users with real world problems when
      using the proposed toolbox.  Researchers and developers would
      also benefit from this perspective.}\\
    As previously mentioned, due to the page limits, we did not added
    any real-world example in the manuscript itself. However, examples
    have been added in the documentation of the
    software. Additionally, we continuously aim at adding real-world
    case studies.
  \item \textit{There is an huge number of methods proposed to address
      the problem of imbalanced domains.
      The proposed toolbox includes several pre-processing methods and two
      algorithm level methods (ensembles). This is on the short side in
      terms of what is currently available.}\\
    The implementation of additional methods is planned for the next
    release of the toolbox and is currently in the issue tracker on
    GitHub. We are striving to include these new methods within the
    next 6 months.
  \item \textit{For instance, an example could be provided for
      studying the impact of using different parameters with a given
      strategy.
      The results could then be presented considering different performance
      assessment metrics.  Given that the issue of performance assessment in
      imbalanced domains is complex,  this example could also show how the
      user may assess the performance in a new metric that is not
      implemented in scikit-learn (for instance G-mean, dominance or Index
      of Balanced Accuracy(IBA)).}\\
    We implemented new metrics available in the \texttt{metrics}
    modules of the toolbox. Sensitivity/specificity, geometric-mean,
    and index of balanced accuracy have been added. Additionally, we
    proposed a method to generate a classification report reporting
    the different metrics. Furthermore, examples presenting the
    metrics have been added to the documentation.
  \item \textit{Another example could also be added considering a
      real world data set. This example could be built to answer the
      question:
      ``What is the best strategy that I can use in this specific
      problem?''} \\
    We provide an example using a real-world dataset which
    illustrate the empirical search for such purpose. A grid-search
    using different balancing method in conjunction with a
    classifier is performed.
  \item \textit{Finally, a third example that considers a problem
      with multiple classes could also be included.} \\
    We introduce an example which illustrate the use of a method for
    a multi-class problem and we additionally generate the according
    classification report.
  \item \textit{With respect to the references to implementations
      available in R, the authors should eventually refer the
      package UBL that is a specific R package for handling
      imbalanced distributions, so it is probably the R package with
      more similar objectives to the current one.} \\
    We refer to this toolbox in the manuscript.
  \end{itemize}


  Thank you for your time and consideration.

  We look forward to your reply.

  \vspace{2\parskip} % Extra whitespace for aesthetics
  \closing{Sincerely,}
  \vspace{2\parskip} % Extra whitespace for aesthetics

  % ----------------------------------------------------------------------------------------

\end{letter}

\end{document}
